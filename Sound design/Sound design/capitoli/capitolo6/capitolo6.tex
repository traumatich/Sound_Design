
\chapter{Sistemi LTI con funzione di trasferimento razionale}

La funzione di trasferimento di un sistema LTI è data da:

\[
H(z) = \frac{B(z)}{A(z)}
\]

\subsection*{Risposta in frequenza}

La risposta in frequenza di un filtro è:

\[
H(e^{j\omega}) = |H(e^{j\omega})| e^{j\phi(\omega)}
\]

\subsection*{Ritardo di gruppo}

Il ritardo di gruppo è dato da:

\[
\tau(\omega) = -\frac{d\phi(\omega)}{d\omega}
\]

\section{Classificazione dei Filtri Digitali}

\subsection*{Filtri a Coefficienti Reali}

- La risposta in frequenza \( H(e^{j\omega}) \) è \textbf{pari}, la fase \( \phi(\omega) \) è \textbf{dispari}.

\subsubsection*{Esempio}

Un esempio di filtro con coefficienti reali è dato da:

\[
H(z) = 1 - a z^{-1}
\]

\subsection*{Filtri a Fase Lineare}

La risposta in frequenza di un filtro a fase lineare è:

\[
H(e^{j\omega}) = c e^{-jK\omega} H_R(\omega)
\]

dove la fase è lineare.

\subsubsection*{Tipi di Filtri a Fase Lineare}

- \textbf{Simmetrico}: fase lineare.
- \textbf{Antisimmetrico}: non adatto a filtri passa-basso o passa-alto.

\section{Specifiche di Progetto}

\subsection*{Banda Passante}

La banda passante è definita come:

\[
0 \leq \omega \leq \omega_p
\]

\subsection*{Banda Oscura}

La banda oscura è definita come:

\[
\omega_s \leq \omega \leq \pi
\]

\subsection*{Zona di Transizione}

La zona di transizione è compresa tra:

\[
\omega_s \quad \text{e} \quad \omega_p
\]

\subsection*{Ampiezza Normalizzata}

L'ampiezza normalizzata è data dalla divisione per:

\[
(1 + \delta_1)
\]

\section{Progetto di Filtri FIR}

\subsection*{Scelta tra FIR e IIR}

- I filtri FIR hanno fase lineare ma un costo computazionale maggiore.
- I filtri IIR sono più efficienti in termini di ordine, ma hanno fase non lineare.

\subsection*{Metodi di Progettazione FIR}

1. \textbf{Windowing}: semplice, ma non ottimizzato.
2. \textbf{Least Square}: minimizza l'errore in frequenza.
3. \textbf{Equiripple}: distribuisce uniformemente l'errore.

\subsubsection*{Metodo Windowing}

Il metodo di windowing prevede il troncamento della risposta impulsiva con una finestra:

- Rettangolare (troncamento semplice).
- Hamming, Blackman, Kaiser (miglior controllo).
- \textbf{Fenomeno di Gibbs}: ripple nella banda passante.

\subsubsection*{Metodo Eigenfilter}

Il metodo eigenfilter minimizza l'errore in banda passante e oscura utilizzando gli autovettori di una matrice.

\subsubsection*{Metodo Equiripple}

L'algoritmo di Parks-McClellan è utilizzato per minimizzare il massimo errore.

\section{Progetto di Filtri IIR}

\subsection*{Trasformazione Bilineare}

1. Progettare il filtro analogico \( H_a(s) \).
2. Applicare la trasformazione bilineare:

\[
s = \frac{1 - z^{-1}}{1 + z^{-1}}
\]

3. Ottenere il filtro digitale \( H(z) \).

\subsection*{Tipologie di Filtri IIR}

1. \textbf{Butterworth}: risposta piatta, transizione lenta.
2. \textbf{Chebyshev I}: banda passante equiripple, banda oscura monotona.
3. \textbf{Chebyshev II}: banda passante monotona, banda oscura equiripple.
4. \textbf{Ellittico}: equiripple in entrambe le bande, transizione più veloce.

\section{Filtri Passa-Tutto}

I filtri passa-tutto mantengono il modulo costante e modificano solo la fase.

\subsection*{Proprietà dei Filtri Passa-Tutto}

- Coppie coniugate reciproche di poli e zeri.
- L'energia di uscita è uguale all'energia di ingresso.

