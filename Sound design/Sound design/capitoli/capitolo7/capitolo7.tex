
\chapter{Introduzione al Suono e Temperamento Musicale}
\section{Suono e Qualità del Suono}

Il suono è costituito da \textbf{vibrazioni} che vengono trasmesse al nostro orecchio tramite un fluido.

\subsection*{Qualità del Suono}

La qualità del suono dipende da vari fattori, tra cui:

\begin{itemize}
    \item \textbf{Altezza}
    \item \textbf{Intensità}
    \item \textbf{Timbro}
    \item \textbf{Durata}
\end{itemize}

\subsection*{Unità di Misura}

L'unità di misura della frequenza del suono è il \textbf{Hertz} (Hz).

\section{Notazione Musicale}

La notazione musicale può essere divisa in due categorie:

\begin{itemize}
    \item \textbf{Notazione assoluta}
    \item \textbf{Codifica relativa}
\end{itemize}

Questa notazione è utilizzata per scrivere, leggere e scambiare la musica in maniera univoca.

\subsection*{Nomi delle Note}

\begin{itemize}
    \item \textbf{Latino}: Do/Ut, Re, Mi, Fa, Sol, La, Si
    \item \textbf{Inglese}: C, D, E, F, G, A, B
    \item \textbf{Alterazioni}: Sharp, Flat, Natural
\end{itemize}

\subsection*{Ottava}

L'ottava è costituita dalla sequenza:

\[
C, C\#, D, D\#, E, F, G, G\#, A, A\#, B, C'
\]

Poi la sequenza si ripete.

\subsection*{Intonazione}

L'intonazione standard è di 440 Hz.

\section{Temperamento Musicale}

Il \textbf{temperamento} si riferisce al rapporto di frequenze relative agli intervalli musicali.

\subsection*{Temperamento Pitagorico}

Il temperamento pitagorico si basa su due principi fondamentali:

\begin{itemize}
    \item \textbf{Ottava}: rapporto di frequenza di \( 2:1 \)
    \item \textbf{Quinta giusta}: rapporto di frequenza di \( 3:2 \)
\end{itemize}

Seguendo il \textbf{circolo delle quinte} (C $\rightarrow$ G $\rightarrow$ D $\rightarrow$ A $\rightarrow$ E $\rightarrow$ ...), dopo 12 quinte si torna a C, ma il rapporto risultante è:

\[
\left(\frac{3}{2}\right)^{12} \approx 129.75
\]

Mentre 7 ottave corrispondono a:

\[
2^7 = 128
\]

Questa discrepanza è nota come \textbf{comma pitagorico}.

\subsubsection*{Soluzione di Pitagora}

Pitagora ha proposto la seguente soluzione:

\begin{itemize}
    \item Mantenere le \textbf{quinte giuste} (3:2) e le \textbf{quarte giuste} (4:3)
    \item Derivare le altre note dal circolo delle quinte e riportarle all'ottava originale
\end{itemize}

In questo modo si ottengono \textbf{intervalli diseguali}, creando difficoltà nell'intonazione in alcune tonalità.

\subsection*{Temperamento Equabile}

Il \textbf{temperamento equabile} è stato sviluppato per risolvere il problema degli intervalli diseguali. In questo sistema, l'ottava è divisa in \textbf{12 intervalli uguali}. 

\begin{itemize}
    \item L'ottava è sempre \( 2:1 \), ma il \textbf{semitono temperato} è calcolato come la dodicesima radice di 2:

    \[
    \sqrt[12]{2} \approx 1.05946
    \]

    \item Questo sistema permette di avere \textbf{tutti gli intervalli identici}, rendendo possibile suonare in qualsiasi tonalità senza dissonanze evidenti.
\end{itemize}

\section{Trasposizione}

\subsection*{Temperamento Equabile}

Nel temperamento equabile, è possibile fare uno "shift" di \( N \) semitoni, in alto o in basso, senza alterare il rapporto di frequenze tra gli intervalli.

\subsection*{Temperamento Pitagorico}

Nel temperamento pitagorico (e in generale nei temperamenti non equabili), questa proprietà non è vera. Una melodia trasposta suonerà in maniera molto differente.

\section{Tempo e Durata}

Il tempo e la durata sono mappati in un tempo relativo, che viene misurato dal valore del metronomo, solitamente indicato come \textbf{BPM} (battiti per minuto).

\subsection*{Battute e Raggruppamenti}

I battiti sono raggruppati in \textbf{battute}.


