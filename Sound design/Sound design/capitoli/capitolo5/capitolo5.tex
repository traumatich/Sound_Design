\chapter{DFT e FFT}
\section*{Sequenze Periodiche}

Un \textbf{evento periodico} è un evento che si ripete uguale a se stesso ad ogni intervallo \( N \).

\begin{itemize}
    \item \(\omega_0 = \frac{2\pi}{N}\) è la \textit{pulsazione}.
    \item Non è né quadraticamente né assolutamente sommabile.
    \item È una convoluzione di un periodo con un treno di impulsi di periodo \( N \).
    \item Può essere rappresentato dalla serie discreta di Fourier (DFS).
\end{itemize}

\textbf{Fattore di Twiddle}: \( W_N^{-kn} \).

\section*{Proprietà della DFS}

\subsection*{Linearità}
\[
\sum_{k=0}^{N-1} c_1 X_1[k] + c_2 X_2[k] W_N^{-kn} = c_1 \sum_{k=0}^{N-1} X_1[k] W_N^{-kn} + c_2 \sum_{k=0}^{N-1} X_2[k] W_N^{-kn}
\]

\subsection*{Dualità}
La DFS di una sequenza periodica è anch'essa periodica.

\subsection*{Traslazione nel Tempo}
\[
x[n - m] \leftrightarrow X(e^{j\omega}) e^{-j\omega m}
\]

\subsection*{Traslazione in Frequenza}
\[
x[n] \leftrightarrow X(e^{j(\omega + \Delta \omega)}) 
\]

\subsection*{Simmetria}
\[
X(e^{-j\omega}) = X^*(e^{j\omega})
\]

\section*{Relazione tra DTFT e DFS}

Eseguire la DFS corrisponde a valutare \( X(z) \) su \( N \) punti equispaziati sul cerchio unitario.

La DTFT di una sequenza periodica è il prodotto delle trasformate di Fourier del singolo periodo e del treno di impulsi:

\[
X(e^{j\omega}) = \sum_{n=-\infty}^{+\infty} x[n] e^{-j\omega n}
\]

\section*{Sequenze Periodiche di Durata Finita}

\begin{itemize}
    \item Una \textbf{sequenza periodica} è completamente rappresentata dalla sua \textbf{sequenza DFS}.
    \item Una sequenza \textbf{aperiodica} di lunghezza \( N \) può essere resa \textbf{periodica} replicandola sull'asse \( n \).
    \item La DFS di una sequenza corrisponde alla DTFT campionata \( N \) in frequenze equispaziate.
\end{itemize}

\[
x[n] \leftrightarrow X(e^{j\omega}) \quad \text{con} \quad \omega = \frac{2\pi}{N}k \quad \text{per} \quad n=0, \dots, N-1, \quad k=0, \dots, N-1.
\]

\section*{Campionamento in Frequenza}

\textbf{Obiettivo}: Determinare le condizioni in cui la conoscenza di campioni equispaziati di \( X(e^{j\omega}) \) permette la ricostruzione univoca della sequenza \( x[n] \).

\subsection*{Principali Trasformate}

\begin{itemize}
    \item \textbf{DTFT} (Trasformata Discreta di Fourier Continua):
    \[
    X(e^{j\omega}) = \sum_{n=-\infty}^{+\infty} x[n] e^{-j\omega n}
    \]
    
    \item \textbf{Campionamento della DTFT in N punti}:
    \[
    \bar{X}[k] = X(e^{j \frac{2\pi}{N} k})
    \]

    \item \textbf{DFS inversa} (per una sequenza periodica di periodo \( N \)):
    \[
    \bar{x}[n] = \frac{1}{N} \sum_{k=0}^{N-1} \bar{X}[k] W_N^{-kn}
    \]
\end{itemize}

\section*{Campionamento in Frequenza e Convoluzione}

Sostituendo la DTFT, otteniamo:

\[
\bar{x}[n] = \frac{1}{N} \sum_{m=-\infty}^{+\infty} x[m] \left( \sum_{k=0}^{N-1} W_N^{-k(n-m)} \right)
\]

Si dimostra che:

\[
\sum_{k=0}^{N-1} W_N^{-k(n-m)} = \sum_{r=-\infty}^{+\infty} \delta[n - m + rN]
\]

Quindi:

\[
\bar{x}[n] = x[n] * \sum_{r=-\infty}^{+\infty} \delta[n + rN]
\]

\textbf{Conclusione}: Il campionamento in frequenza della DTFT è equivalente a replicare periodicamente la sequenza nel dominio temporale.

\section{Condizioni per la ricostruzione univoca}

Definiamo:

\[
\tilde{y}[n] = \sum_{r=-\infty}^{+\infty} \delta[n + rN]
\]

La sua \textbf{trasformata discreta} è:

\[
\tilde{Y}[k] = \sum_{n=0}^{N-1} \delta[n] e^{-j \left( \frac{2\pi}{N} \right) k n} = 1 \quad \text{per} \quad k = 0, \dots, N-1
\]

Invertendo la DFS (Trasformata di Fourier Discreta):

\[
\tilde{y}[n] = \frac{1}{N} \sum_{k=0}^{N-1} 1 \cdot W_N^{-kn} = \sum_{r=-\infty}^{+\infty} \delta[n + rN]
\]

\textbf{Conclusione:}

Il \textbf{campionamento in frequenza} della DTFT implica che la sequenza venga \textbf{replicata periodicamente} nel dominio temporale.

Se la sequenza originale è limitata a \(N\) campioni, la sua conoscenza in \(N\) punti equispaziati della DTFT è sufficiente per la ricostruzione univoca.


\section*{Aliasing nel Tempo}

Campionare \( X(e^{j\omega}) \) in \( N \) punti equispaziati sul cerchio unitario equivale ad avere la DFS di una sequenza periodica di periodo \( N \). Questa sequenza periodica si ottiene facendo la convoluzione di \( x[n] \) con un treno di impulsi spaziati di \( N \).

\[
\bar{x}[n] = x[n] * \sum_{r=-\infty}^{+\infty} \delta[n + rN]
\]

\textbf{1° Caso}: Se \( x[n] \) ha durata finita \( \leq N \), può essere ricostruita perfettamente dai suoi \( N \) campioni.

\textbf{2° Caso}: Se \( x[n] \) ha durata maggiore di \( N \), si verifica aliasing temporale.

\section*{Trasformata Discreta di Fourier (DFT)}

Una sequenza aperiodica \( x[n] \) di lunghezza \( N \) può essere completamente rappresentata da \( N \) campioni della sua DTFT. La DFT di una sequenza finita di lunghezza \( N \) è anch'essa di lunghezza \( N \).

\textbf{DFT Diretta}:

\[
X[k] = \sum_{n=0}^{N-1} x[n] W_N^{kn}
\]

\textbf{DFT Inversa}:

\[
x[n] = \frac{1}{N} \sum_{k=0}^{N-1} X[k] W_N^{-kn}
\]

\section{Zero Padding}

\textbf{Obiettivo}: Aumentare la lunghezza della DFT senza alterarne il contenuto informativo.

\subsection{Relazione tra DFT e DTFT}
La \textbf{DFT} di \( x[n] \) è ottenuta campionando la \textbf{DTFT} di \( x[n] \) nei punti:

\[
X[k] = \sum_{n=0}^{N-1} x[n] e^{-j \frac{2\pi}{N} k n}
\]

Per \textbf{ricostruire perfettamente} \( x[n] \), servono \textbf{almeno N campioni} della DTFT per evitare aliasing temporale.

\subsection{Effetto dello Zero Padding}
Si \textbf{aumenta la lunghezza della DFT} a \( M > N \) (maggiore risoluzione in frequenza).
Questo si ottiene \textbf{aggiungendo} \( M - N \) zeri alla fine di \( x[n] \):

\[
x_{padded}[n] = \begin{cases}
x[n] & \text{per} \quad 0 \leq n \leq N-1 \\
0 & \text{per} \quad N \leq n \leq M-1
\end{cases}
\]

La \textbf{DFT non cambia}, ma si ottengono \textbf{più campioni della DTFT}, quindi una \textbf{migliore interpolazione} della risposta in frequenza.

\subsection{Effetti dello Zero Padding}
\begin{itemize}
    \item Aumenta la \textbf{risoluzione in frequenza} (più punti disponibili)
    \item \textbf{Non altera il contenuto in frequenza} (non aggiunge nuove informazioni)
    \item Utile per \textbf{visualizzare meglio} la trasformata
\end{itemize}


\section{Convoluzione di Sequenze Periodiche}

Se \( \tilde{x}_1[n] \) e \( \tilde{x}_2[n] \) sono \textbf{sequenze periodiche di periodo N}, allora:

\subsection{Convoluzione Periodica (Proprietà Commutativa)}
\[
y[n] = \sum_{m=0}^{N-1} \tilde{x}_1[m] \cdot \tilde{x}_2[n-m \, (\text{mod} \, N)]
\]

\subsection{Relazione con la DFS}
\[
\tilde{X}_1[n] \cdot \tilde{X}_2[n] = \mathcal{DFT}\left\{ \tilde{x}_1[n] \cdot \tilde{x}_2[n] \right\}
\]

\subsection{Moltiplicazione punto a punto nel dominio del tempo}
La moltiplicazione punto a punto nel dominio del tempo equivale a una convoluzione nel dominio della DFS:
\[
    \tilde{X}_1[n] \cdot \tilde{X}_2[n] = \sum_{m=0}^{N-1} \tilde{x}_1[m] \cdot \tilde{x}_2[n - m \, (\text{mod} \, N)]
\]

\subsection{Convoluzione Circolare}
\textbf{Definizione}: Identica alla convoluzione periodica, ma gli indici sono considerati \textbf{modulo N}.

Per due sequenze finite di lunghezza \( N \):
\[
    X_1[n] \cdot X_2[n] = \sum_{m=0}^{N-1} x_1[m] \cdot x_2[(n - m) \, (\text{mod} \, N)]
\]

\subsection{Relazione con la DFT}
\[
    X_1[n] \cdot X_2[n]= X_1[k] \cdot X_2[k]
\]

\subsection{Moltiplicazione punto a punto nel tempo}
\[
    X_1[n] \cdot X_2[n] = \sum_{m=0}^{N-1} x_1[m] \cdot x_2[(n - m) \, (\text{mod} \, N)]
\]

\section{Rapporto tra Convoluzione Aperiodica e Circolare}

\subsection*{Caso 1: Sequenze finite di lunghezza \( N \)}
\[
\begin{array}{|c|c|}
\hline
\textbf{Aperiodica} & \textbf{Circolare} \\
\hline
x_1[n] * x_2[n] = \sum_{m=0}^{N-1} x_1[m] \cdot x_2[n-m] & x_1[n] * x_2[n] = \sum_{m=0}^{N-1} x_1[m] \cdot x_2[(n - m) \, (\text{mod} \, N)] \\
\hline
\text{DTFT: } X_1(e^{j\omega}) \cdot X_2(e^{j\omega}) & \text{DFT: } X_1[k] \cdot X_2[k] \\
\hline
\text{Risultato: sequenza aperiodica di lunghezza } 2N-1 & \text{Risultato: sequenza periodica di lunghezza N} \\
\hline
\text{Non coincidono in generale} & \\
\hline
\end{array}
\]

\subsection*{Caso 2: Sequenze di lunghezza diversa}
\begin{itemize}
    \item \textbf{Sequenza aperiodica di lunghezza L}
    \item \textbf{Sequenza aperiodica di lunghezza P < L}
\end{itemize}
\[
\begin{array}{|c|c|}
\hline
\textbf{Aperiodica} & \textbf{Circolare} \\
\hline
x_1[n] * x_2[n] = \sum_{m=0}^{L-1} x_1[m] \cdot x_2[n-m] & x_1[n] * x_2'[n] = \sum_{m=0}^{L-1} x_1[m] \cdot x_2'[(n - m) \, (\text{mod} \, L)] \\
\hline
\text{Risultato: sequenza aperiodica di lunghezza } L + P - 1 & \text{Risultato: sequenza circolare di lunghezza L} \\
\hline
& x_2'[n] \text{ è } x_2[n] \text{ con zero-padding fino a } L-1 \\
\hline
\end{array}
\]

\section{Implementazione della Convoluzione Circolare}
\begin{enumerate}
    \item Si \textbf{forma la sequenza}.
    \item Si \textbf{trasla circolarmente}.
    \item Si \textbf{moltiplica punto a punto}.
\end{enumerate}

\section{Costo Computazionale}

Per una sequenza di lunghezza \( N \):

\begin{itemize}
    \item \textbf{Moltiplicazioni complesse} \( \Rightarrow N^2 \) (equivalente a \( 4N^2 \) operazioni reali)
    \item \textbf{Sommatorie complesse} \( \Rightarrow N(N-1) \) (equivalente a \( N(4N - 2) \) operazioni reali)
\end{itemize}

\textbf{Riduzione del costo computazionale}

\begin{itemize}
    \item \textbf{Algoritmi ottimizzati} per il calcolo della DFT a costo inferiore a \( O(N^2) \)
    \item \textbf{Esempio}: Fast Fourier Transform (FFT)
\end{itemize}


\section{Fast Fourier Transform (FFT)}

L'FFT sfrutta proprietà matematiche per ridurre il numero di operazioni necessarie:

\subsection{Formula della DFT}
\[
X[k] = \sum_{n=0}^{N-1} x[n] \cdot e^{-j \frac{2\pi}{N} k n}
\]

Per ogni campione della DFT:
\begin{itemize}
    \item \textbf{Moltiplicazioni complesse}: \( N \)
    \item \textbf{Somme complesse}: \( N-1 \)
\end{itemize}

Per calcolare \( N \) campioni:
\begin{itemize}
    \item Complessità totale: \( O(N^2) \) \quad \text{(ALTA!)}
\end{itemize}

\subsection{\texorpdfstring{Ottimizzazioni basate sulle Proprietà di \( W_N \)}{Ottimizzazioni basate sulle Proprietà di W\_N}}


\begin{itemize}
    \item \textbf{Simmetria}
    \item \textbf{Periodicità}
\end{itemize}

\section{Radix-2 Decimation In Time (DIT) FFT}

\textbf{Idea (Cooley-Tukey 1965)}

\begin{itemize}
    \item Si suddivide la sequenza \( x[n] \) in due sottosequenze di lunghezza \( N/2 \):
    \begin{itemize}
        \item Campioni \textbf{con indice pari}
        \item Campioni \textbf{con indice dispari}
    \end{itemize}
    \item Si calcola la \textbf{DFT totale} componendo opportunamente le \textbf{DFT delle sottosequenze}.
\end{itemize}

\textbf{Scomposizione della DFT}

\[
X[k] = \sum_{m=0}^{N/2-1} x[2m] W_{N/2}^{km} + W_N^k \sum_{m=0}^{N/2-1} x[2m+1] W_{N/2}^{km}
\]

La DFT a \( N \) punti si scompone in due DFT a \( N/2 \) punti più combinazioni di dati.

\textbf{Algoritmo Iterativo}

\begin{itemize}
    \item Se \( N \) è una potenza di 2, l'algoritmo si applica ricorsivamente fino alla DFT di 2 punti.
    \item \textbf{Numero di stadi}: \( M = \log_2(N) \)
    \item \textbf{Numero totale di "butterfly" per stadio}: \( N/2 \)
\end{itemize}

\textbf{Esempio per \( N=8 \):}

\[
N = 2^3 \Rightarrow 3 \text{ stadi}: \quad \text{DFT 8 punti} \rightarrow \text{DFT 4 punti} \rightarrow \text{DFT 2 punti}
\]
\section*{Varianti dell'algoritmo DIT}

\begin{enumerate}
    \item \textbf{DIT Canonica}
    \begin{itemize}
        \item Ingresso: \textbf{bit-reversed order}
        \item Uscita: \textbf{ordine normale}
        \item \textbf{"In-place"}
    \end{itemize}
    \item \textbf{DIT Alternativa}
    \begin{itemize}
        \item Ingresso: \textbf{ordine normale}
        \item Uscita: \textbf{bit-reversed order}
        \item \textbf{"In-place"}
    \end{itemize}
    \item \textbf{DIT Sequenziale}
    \begin{itemize}
        \item Indici normali in ogni stadio
        \item Non \textbf{in-place} (più complesso)
    \end{itemize}
\end{enumerate}

\section*{Radix-2 Decimation In Frequency (DIF) FFT}

\textbf{Idea:}

La sequenza in ingresso viene divisa in due sottosequenze:
\begin{itemize}
    \item Campioni con indice pari
    \item Campioni con indice dispari
\end{itemize}

Si calcolano separatamente e poi si combinano in uscita.

\section*{Scomposizione della DFT}

\begin{itemize}
    \item Per i campioni \textbf{pari}:
    \[
    X[2m] = \sum_{n=0}^{N/2-1} \left[ x[n] + x[n+N/2] \right] W_{N/2}^{nm}
    \]
    \item Per i campioni \textbf{dispari}:
    \[
    X[2m+1] = \sum_{n=0}^{N/2-1} \left[ x[n] - x[n+N/2] \right] W_{N/2}^{nm} W_N^n
    \]
\end{itemize}

\section*{Algoritmo Iterativo}

\begin{itemize}
    \item Se \( N = 2^M \), si applica ricorsivamente fino a ottenere \textbf{solo DFT a 2 punti}.
    \item \textbf{Numero di stadi}: 
    \[
    M = \log_2 (N)
    \]
    \item \textbf{Numero di "butterfly" per stadio}: \( \frac{N}{2} \)
\end{itemize}

\section*{Butterfly DIF}

\begin{itemize}
    \item Struttura simile a DIT, ma con \textbf{dati e flussi invertiti}.
    \item \textbf{"In-place"} possibile con un solo array di memoria.
\end{itemize}
\section*{Differenze tra DIT e DIF}

\begin{center}
\begin{tabular}{|c|c|c|}
\hline
\textbf{Caratteristica} & \textbf{DIT (Decimation In Time)} & \textbf{DIF (Decimation In Frequency)} \\
\hline
Scomposizione & Sui \textbf{campioni di ingresso} & Sui \textbf{coefficienti di uscita} \\
\hline
Bit-Reversal & \textbf{Ingresso in bit-reversed order} & \textbf{Uscita in bit-reversed order} \\
\hline
Ordine dei W & Normale & Bit-reversed \\
\hline
Algoritmo & \textbf{Top-down} & \textbf{Bottom-up} \\
\hline
\end{tabular}
\end{center}

\section*{Radix-2 Decimation in Time (DIT) FFT}

\textbf{Principio $\rightarrow$} Si scompone la sequenza \( x[n] \) in due sottosequenze (pari e dispari).

\subsection*{Procedura}
\begin{enumerate}
    \item \textbf{Scomposizione ricorsiva} fino a ottenere DFT di lunghezza 2.
    \item \textbf{Combinazione dei risultati} con operazioni butterfly.
    \item \textbf{Bit-reversal} sugli indici in ingresso o uscita.
\end{enumerate}

\section*{Struttura della DIT FFT}

\textbf{Ingresso}: Ordinamento \textbf{bit-reversed}

\textbf{Uscita}: Ordinamento normale

\section*{Butterfly}
\[
X[k] = X'[k] + W_N^k X''[k]
\]

\section*{Computazione in-place}
Un unico array aggiornato a ogni stadio.
\section*{DIT Canonica}

\begin{itemize}
    \item \textbf{Input}: Bit-reversed
    \item \textbf{Output}: Normale
    \item \textbf{Coefficienti W}: Normali
\end{itemize}

\section*{DIT Alternativa}

\begin{itemize}
    \item \textbf{Input}: Normale
    \item \textbf{Output}: Bit-reversed
    \item \textbf{Coefficienti W}: Bit-reversed
\end{itemize}

\section*{DIT Sequenziale}

\begin{itemize}
    \item Indici normali a ogni stadio
    \item \textbf{Non in-place} (maggior uso di memoria)
\end{itemize}

\section*{Struttura della DIF FFT}

\begin{itemize}
    \item \textbf{Ingresso}: Ordinamento normale
    \item \textbf{Uscita}: Ordinamento \textbf{bit-reversed}
\end{itemize}

\section*{Butterfly}
\[
X[k] = X_e[k] + W_N^k X_o[k]
\]

Computazione in-place

\section*{DIF Canonica}

\begin{itemize}
    \item \textbf{Input}: Normale
    \item \textbf{Output}: Bit-reversed
    \item \textbf{Coefficienti W}: Normali
\end{itemize}

\section*{DIF Alternativa}

\begin{itemize}
    \item \textbf{Input}: Bit-reversed
    \item \textbf{Output}: Normale
    \item \textbf{Coefficienti W}: Bit-reversed
\end{itemize}

\section*{DIF Sequenziale}

\begin{itemize}
    \item Indici normali a ogni stadio
    \item \textbf{Non in-place}
\end{itemize}

\section*{Confronto DIT vs DIF}

\begin{center}
\begin{tabular}{|c|c|c|}
\hline
\textbf{Caratteristica} & \textbf{DIT FFT} & \textbf{DIF FFT} \\
\hline
\textbf{Divisione} & Ingresso (x[n]) & Uscita (X[k]) \\
\hline
\textbf{Input} & Bit-reversed & Normale \\
\hline
\textbf{Output} & Normale & Bit-reversed \\
\hline
\textbf{Computazione} & Bottom-up & Top-down \\
\hline
\textbf{Uso memoria} & In-place & In-place \\
\hline
\end{tabular}
\end{center}

\section*{Costo Computazionale}

\begin{itemize}
    \item \textbf{Butterfly}: 1 prodotto complesso + 2 somme complesse.
    \item \textbf{N punti}: N/2 butterfly per stadio.
    \item \textbf{FFT a N punti} ($N = 2^M$): \textbf{M stadi}.
\end{itemize}

\section*{Confronto DFT vs FFT}

\begin{itemize}
    \item \textbf{DFT classica}: O($N^2$)
    \item \textbf{FFT radix-2}: O($N \log N$)
\end{itemize}

\section*{Vantaggi della FFT}

\begin{itemize}
    \item \textbf{Enorme riduzione dei calcoli} rispetto alla DFT.
    \item \textbf{Hardware ottimizzato}: pipeline, unità parallele.
    \item \textbf{Utilizzata in DSP, compressione, imaging}.
\end{itemize}

\section*{Problema principale}

\begin{itemize}
    \item \textbf{Precisione finita nei calcoli numerici.}
\end{itemize}

\section*{Categorie principali di algoritmi FFT}

Gli \textbf{algoritmi FFT} sono stati sviluppati per velocizzare il calcolo della Trasformata Discreta di Fourier (DFT).

\subsection*{Categorie}

\begin{itemize}
    \item \textbf{Radix-M FFT (CFA – Common Factor Algorithm)}:
    \begin{itemize}
        \item $N$ deve essere una potenza di $M$. Il caso più diffuso è il \textbf{radix-2}.
        \item Altri esempi includono \textbf{radix-4} o \textbf{radix-8}, che riducono il numero di operazioni richieste.
    \end{itemize}

    \item \textbf{Mixed Radix FFT}:
    \begin{itemize}
        \item Scompone $N$ come un prodotto di fattori diversi (tipicamente 2, 3, 4 o 8).
        \item Ogni stadio dell'algoritmo utilizza un radix differente, migliorando l’efficienza del calcolo rispetto all’approccio standard.
    \end{itemize}

    \item \textbf{Prime Factor FFT (PFA – Prime Factor Algorithm)}:
    \begin{itemize}
        \item $N$ viene scomposto in fattori primi tra loro.
        \item Riduce ulteriormente il numero di moltiplicazioni rispetto alla Mixed Radix FFT.
    \end{itemize}

    \item \textbf{Algoritmo di Winograd}:
    \begin{itemize}
        \item Sfrutta una struttura matematica per trasformare la FFT monodimensionale in una FFT bidimensionale.
        \item Il costo computazionale complessivo è $O(N \log_2 N)$.
        \item Riduce il numero di moltiplicazioni a $O(N)$, seppur aumentando il numero di somme da effettuare.
    \end{itemize}
\end{itemize}

\section*{Utilizzi}

\begin{itemize}
    \item \textbf{Progettazione di filtri FIR}: La trasformata inversa di Fourier (IFFT) viene utilizzata per ottenere il filtro desiderato a partire da una griglia di punti in frequenza.
    \item \textbf{Calcolo della convoluzione e della correlazione}: La FFT permette di eseguire queste operazioni in maniera molto più efficiente rispetto ai metodi tradizionali.
    \item \textbf{Stima spettrale}: Viene sfruttata la relazione tra la DFT e la Trasformata di Fourier Discreta Tempo-Variabile (DTFT) per analizzare segnali di durata finita.
    \item \textbf{Riconoscimento vocale e sistemi radar}: Questi settori utilizzano la FFT per analizzare segnali il cui contenuto spettrale cambia nel tempo.
\end{itemize}

\textbf{Problema critico} → \textbf{Precisione finita}, poiché i calcoli numerici possono introdurre errori dovuti alla rappresentazione discreta dei numeri in hardware.

\textbf{Elaborazione numerica dei segnali (DSP)}: La FFT continua a essere una delle tecniche più utilizzate in innumerevoli applicazioni tecnologiche.

\section*{Calcolo della Convoluzione e della Correlazione con la FFT}

\subsection*{Convoluzione Aperiodica e FFT}

Per due sequenze di lunghezza \( L \) e \( P \), la convoluzione aperiodica può essere calcolata tramite una convoluzione ciclica di \( L+P-1 \) punti. Per sfruttare l'efficienza della FFT, si utilizza lo \textbf{zero-padding} fino alla lunghezza \( L+P-1 \).

\subsection*{Formula della Convoluzione}

La convoluzione aperiodica di due segnali \( x_1[n] \) e \( x_2[n] \) è definita come:

\[
x_1[n] * x_2[n] = \sum_{m=-\infty}^{+\infty} x_1[m] \cdot x_2[n-m]
\]

Questa relazione nel dominio della trasformata diventa:

\[
X_1(e^{j\omega}) \cdot X_2(e^{j\omega})
\]

\subsection*{Schema del Calcolo della Convoluzione con la FFT}

\begin{enumerate}
    \item \textbf{Zero-padding} delle sequenze fino alla lunghezza \( L+P-1 \).
    \item \textbf{Calcolo della FFT} delle sequenze con la stessa lunghezza.
    \item \textbf{Moltiplicazione} delle FFT ottenute.
    \item \textbf{Calcolo della IFFT} per ottenere la convoluzione.
\end{enumerate}

\subsection*{Formula della Correlazione}

La correlazione tra due segnali è definita come:

\[
x_1[n] * x_2[-n] = \sum_{m=-\infty}^{+\infty} x_1[m] \cdot x_2[n+m]
\]

E nel dominio della trasformata diventa:

\[
X_1(e^{j\omega}) \cdot X_2^*(e^{j\omega})
\]

dove \( X_2^* \) è il complesso coniugato della FFT di \( x_2[n] \).

\subsection*{Schema del Calcolo della Correlazione con la FFT}

\begin{enumerate}
    \item \textbf{Zero-padding} delle sequenze fino alla lunghezza \( L+P-1 \).
    \item \textbf{Calcolo della FFT} delle sequenze.
    \item \textbf{Moltiplicazione della FFT di \( x_1[n] \)} con il coniugato complesso della FFT di \( x_2[n] \).
    \item \textbf{Calcolo della IFFT} per ottenere la correlazione.
\end{enumerate}

\subsection*{Ottimizzazioni e Speed-up}

\begin{itemize}
    \item \textbf{Bit-Reversing Ottimizzato:} Si può evitare il bit-reversing usando un algoritmo \textbf{DIF diretto} per la FFT e \textbf{DIT inverso} per la IFFT.
    \item \textbf{Calcolo di DFT in parallelo:} È possibile calcolare \textbf{due DFT di sequenze reali contemporaneamente}, sfruttando la proprietà della FFT per segnali reali.
\end{itemize}

\subsection*{Se una delle Sequenze ha Durata Molto Lunga (Infinita)}

\begin{itemize}
    \item \textbf{Overlap and Add:} Si decompone \( x[n] \) in blocchi senza sovrapposizione.
    \item \textbf{Overlap and Save:} Si decompone \( x[n] \) in blocchi con sovrapposizione di \( P-1 \) punti, calcolando la convoluzione circolare di ogni blocco (FFT a \( L \) punti).
\end{itemize}

