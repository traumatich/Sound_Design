

\chapter{Formule pratiche per i coefficienti dei filtri EQ biquad}

Tutte le funzioni di trasferimento dei filtri sono derivate da prototipi analogici (che sono mostrati sotto per ciascun tipo di filtro EQ) e sono state digitalizzate usando la Trasformata Bilineare (Bilinear Transform). La distorsione di frequenza (frequency warping) della BLT è stata presa in considerazione sia per la rilocalizzazione della frequenza (il normale "prewarping") sia per la ri-regolazione della larghezza di banda (poiché essa viene compressa nella trasformazione da analogico a digitale).

Partiamo da una funzione di trasferimento biquad definita come:
\begin{equation}
H(z) = \frac{b_0 + b_1 z^{-1} + b_2 z^{-2}}{a_0 + a_1 z^{-1} + a_2 z^{-2}} \tag{Eq 1}
\end{equation}

Questo mostra 6 coefficienti invece di 5 quindi, a seconda della tua architettura, potresti normalizzare $a_0$ a 1 e forse anche $b_0$ a 1 (e raccogliere questo in un coefficiente di guadagno complessivo). Allora la funzione di trasferimento diventa:

\begin{equation}
H(z) = \frac{\frac{b_0}{a_0} + \frac{b_1}{a_0} z^{-1} + \frac{b_2}{a_0} z^{-2}}{1 + \frac{a_1}{a_0} z^{-1} + \frac{a_2}{a_0} z^{-2}} \tag{Eq 2}
\end{equation}

oppure

\begin{equation}
H(z) = \left(\frac{b_0}{a_0}\right) \cdot \frac{1 + \frac{b_1}{b_0} z^{-1} + \frac{b_2}{b_0} z^{-2}}{1 + \frac{a_1}{a_0} z^{-1} + \frac{a_2}{a_0} z^{-2}} \tag{Eq 3}
\end{equation}

L'implementazione più diretta sarebbe la "Forma Diretta 1" (Eq 2):

\begin{equation}
y[n] = \frac{b_0}{a_0} x[n] + \frac{b_1}{a_0} x[n-1] + \frac{b_2}{a_0} x[n-2] - \frac{a_1}{a_0} y[n-1] - \frac{a_2}{a_0} y[n-2] \tag{Eq 4}
\end{equation}

\section*{Parametri definiti dall'utente}

\begin{itemize}
    \item $F_s$: frequenza di campionamento
    \item $f_0$: frequenza centrale o di taglio (o di metà guadagno per shelf)
    \item dBgain: solo per i filtri peaking e shelving
    \item Q: definizione tipica ingegneristica (con aggiustamento per il filtro peakingEQ)
    \item oppure BW: larghezza di banda in ottave
    \item oppure S: parametro di pendenza per gli shelving
\end{itemize}

\section*{Variabili intermedie}

\begin{align*}
A &= \sqrt{10^{\text{dBgain}/20}} = 10^{\text{dBgain}/40} \
\omega_0 &= 2\pi f_0 / F_s \\
\alpha &= \begin{cases}
    \frac{\sin(\omega_0)}{2Q} & \text{(caso Q)} \\
    \sin(\omega_0) \sinh \left( \frac{\ln(2)}{2} \cdot \text{BW} \cdot \frac{\omega_0}{\sin(\omega_0)} \right) & \text{(caso BW)} \\
    \frac{\sin(\omega_0)}{2} \sqrt{(A + 1/A)(1/S - 1) + 2} & \text{(caso S)}
\end{cases}
\end{align*}

\section*{Coefficienti per ciascun tipo di filtro}

\subsection*{Passa basso (LPF)}
\begin{align*}
b_0 &= \frac{1 - \cos(\omega_0)}{2} \\
b_1 &= 1 - \cos(\omega_0) \\
b_2 &= \frac{1 - \cos(\omega_0)}{2} \\
a_0 &= 1 + \alpha \\
a_1 &= -2 \cos(\omega_0) \\
a_2 &= 1 - \alpha
\end{align*}

\subsection*{Passa alto (HPF)}
\begin{align*}
b_0 &= \frac{1 + \cos(\omega_0)}{2} \\
b_1 &= -(1 + \cos(\omega_0)) \\
b_2 &= \frac{1 + \cos(\omega_0)}{2} \\
a_0 &= 1 + \alpha \\
a_1 &= -2 \cos(\omega_0) \\
a_2 &= 1 - \alpha
\end{align*}

\subsection*{Passa banda (BPF) - guadagno costante nella skirt}
\begin{align*}
b_0 &= \frac{\sin(\omega_0)}{2} = Q\alpha \\
b_1 &= 0 \\
b_2 &= -\frac{\sin(\omega_0)}{2} = -Q\alpha \\
a_0 &= 1 + \alpha \\
a_1 &= -2 \cos(\omega_0) \\
a_2 &= 1 - \alpha
\end{align*}

\subsection*{Passa banda (BPF) - guadagno di picco a 0 dB}
\begin{align*}
b_0 &= \alpha \\
b_1 &= 0 \\
b_2 &= -\alpha \\
a_0 &= 1 + \alpha \\
a_1 &= -2 \cos(\omega_0) \\
a_2 &= 1 - \alpha
\end{align*}

\subsection*{Notch}
\begin{align*}
b_0 &= 1 \\
b_1 &= -2 \cos(\omega_0) \\
b_2 &= 1 \\
a_0 &= 1 + \alpha \\
a_1 &= -2 \cos(\omega_0) \\
a_2 &= 1 - \alpha
\end{align*}

\subsection*{Filtro all-pass (APF)}
\begin{align*}
b_0 &= 1 - \alpha \\
b_1 &= -2 \cos(\omega_0) \\
b_2 &= 1 + \alpha \\
a_0 &= 1 + \alpha \\
a_1 &= -2 \cos(\omega_0) \\
a_2 &= 1 - \alpha
\end{align*}

\subsection*{Peaking EQ}
\begin{align*}
b_0 &= 1 + \alpha A \\
b_1 &= -2 \cos(\omega_0) \\
b_2 &= 1 - \alpha A \\
a_0 &= 1 + \alpha / A \\
a_1 &= -2 \cos(\omega_0) \\
a_2 &= 1 - \alpha / A
\end{align*}

\subsection*{Low Shelf}
\begin{align*}
b_0 &= A\left[(A+1) - (A-1)\cos(\omega_0) + 2\sqrt{A}\alpha\right] \\
b_1 &= 2A\left[(A-1) - (A+1)\cos(\omega_0)\right] \\
b_2 &= A\left[(A+1) - (A-1)\cos(\omega_0) - 2\sqrt{A}\alpha\right] \\
a_0 &= (A+1) + (A-1)\cos(\omega_0) + 2\sqrt{A}\alpha \\
a_1 &= -2\left[(A-1) + (A+1)\cos(\omega_0)\right] \\
a_2 &= (A+1) + (A-1)\cos(\omega_0) - 2\sqrt{A}\alpha
\end{align*}

\subsection*{High Shelf}
\begin{align*}
b_0 &= A\left[(A+1) + (A-1)\cos(\omega_0) + 2\sqrt{A}\alpha\right] \\
b_1 &= -2A\left[(A-1) + (A+1)\cos(\omega_0)\right] \\
b_2 &= A\left[(A+1) + (A-1)\cos(\omega_0) - 2\sqrt{A}\alpha\right] \\
a_0 &= (A+1) - (A-1)\cos(\omega_0) + 2\sqrt{A}\alpha \\
a_1 &= 2\left[(A-1) - (A+1)\cos(\omega_0)\right] \\
a_2 &= (A+1) - (A-1)\cos(\omega_0) - 2\sqrt{A}\alpha
\end{align*}

\section*{Trasformata Bilineare e Sostituzioni}

La trasformata bilineare (con compensazione della distorsione di frequenza) effettua la seguente sostituzione:

\[
s \leftarrow \frac{1}{\tan(\omega_0/2)} \cdot \frac{1 - z^{-1}}{1 + z^{-1}}
\]

Utilizzando le seguenti identità trigonometriche:

\[
\tan(\omega_0/2) = \frac{\sin(\omega_0)}{1 + \cos(\omega_0)} \qquad \text{e} \qquad \tan^2(\omega_0/2) = \frac{1 - \cos(\omega_0)}{1 + \cos(\omega_0)}
\]

otteniamo le seguenti sostituzioni:

\[
1 \leftarrow \frac{1 + \cos(\omega_0)}{1 + \cos(\omega_0)} \cdot \frac{1 + 2z^{-1} + z^{-2}}{1 + 2z^{-1} + z^{-2}}
\]

\[
s \leftarrow \frac{1 + \cos(\omega_0)}{\sin(\omega_0)} \cdot \frac{1 - z^{-1}}{1 + z^{-1}} = \frac{1 + \cos(\omega_0)}{\sin(\omega_0)} \cdot \frac{1 - z^{-2}}{1 + 2z^{-1} + z^{-2}}
\]

\[
s^2 \leftarrow \frac{1 + \cos(\omega_0)}{1 - \cos(\omega_0)} \cdot \frac{1 - 2z^{-1} + z^{-2}}{1 + 2z^{-1} + z^{-2}}
\]

Il fattore:

\[
\frac{1 + \cos(\omega_0)}{1 + 2z^{-1} + z^{-2}}
\]

è comune a tutti i termini sia al numeratore che al denominatore, può quindi essere raccolto e successivamente eliminato dalle sostituzioni, portando a:

\[
1 \leftarrow \frac{1 + 2z^{-1} + z^{-2}}{1 + \cos(\omega_0)}
\]

\[
s \leftarrow \frac{1 - z^{-2}}{\sin(\omega_0)}
\]

\[
s^2 \leftarrow \frac{1 - 2z^{-1} + z^{-2}}{1 - \cos(\omega_0)}
\]

Inoltre, tutti i termini (sia numeratore che denominatore) possono essere moltiplicati per un fattore comune $\sin^2(\omega_0)$, risultando infine nelle seguenti sostituzioni:

\[
1 \leftarrow (1 + 2z^{-1} + z^{-2}) \cdot (1 - \cos(\omega_0))
\]

\[
s \leftarrow (1 - z^{-2}) \cdot \sin(\omega_0)
\]

\[
s^2 \leftarrow (1 - 2z^{-1} + z^{-2}) \cdot (1 + \cos(\omega_0))
\]

\[
1 + s^2 \leftarrow 2 \cdot (1 - 2\cos(\omega_0)z^{-1} + z^{-2})
\]

Le formule dei coefficienti biquad derivano dalle espressioni sopra dopo una piccola semplificazione.


