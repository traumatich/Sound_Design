\chapter{Effetti basati su DL}

\section{Lista di effetti basati su Delay Line}
\begin{itemize}
    \item Digital Delay
    \item Reverb
    \item Chorus
    \item Flanger
    \item Vibrato
\end{itemize}

\section{Delay Line}
\textbf{• Con il termine Delay Line si intende un buffer di memoria lungo Nsamples nel quale viene scritto un dato e successivamente viene letto}
\textbf{• Time = distanza tra i puntatori di scrittura e lettura}

\section{Vibrato}
\textbf{Il dato viene letto a diverse velocità così da cambiarne il pitch}

\section{Slapback/Echo/Delay}
\begin{itemize}
    \item \textbf{Slapback}(o doubling): 10 < M < 25 ms
    \item \textbf{Echo}: M > 50 ms
    \item g setta il dry/wet
\end{itemize}

\begin{itemize}
    \item \textbf{Delay}(simula multiple riflessioni)
    \item g setta il dry/wet
\end{itemize}

\section{Flanger/Chorus}
\begin{itemize}
    \item \textbf{Flanger}: M < 15 ms, variabile periodicamente (LFO)
    \item \textbf{Chorus}: generato da copie del segnale a pitch leggermente modificati: si usa il delay che genera più vibrato, sommati al segnale diretto
    \item M1, M2 10-25 ms, diversi tra loro
\end{itemize}

\section{Phaser}
\begin{itemize}
    \item L'orecchio umano non è molto sensibile alle differenze di fase, ma questo crea interferenze udibili quando viene mixato con il segnale dry (non processato), creando dei notch.
    \item Il numero di filtri all-pass (solitamente chiamati stadi) varia a seconda dei modelli; alcuni phaser analogici offrono 4, 6, 8 o 12 stadi.
\end{itemize}

\section*{Riverbero}

\begin{itemize}
    \item La riverberazione naturale è una condizione tipica degli ambienti spazialmente delimitati (stanze, auditorium, chiese, teatri, ecc).
    
    \item Ogni ambiente ha una propria e caratteristica riverberazione (si pensi al tipico riverbero di una chiesa o di un cinema).
    
    \item In natura, gli ambienti privi di riverbero sono una condizione molto difficile da avere.
    
    \item Esistono apposite camere dette anecoiche, le quali sono studiate per eliminare le riverberazioni ambientali.
\end{itemize}

\begin{itemize}
    \item In rosso: 
    \begin{itemize}
        \item Il path diretto dalla sorgente all’ascoltatore.
    \end{itemize}
    
    \item In blu:
    \begin{itemize}
        \item Le riflessioni ambientali (riverbero).
    \end{itemize}
    
    \item Esistono diverse tipologie di riflessioni a seconda di quanti «rimbalzi» ci sono tra sorgente e ascoltatore.
\end{itemize}

\section{Room Impulse Response (RIR)}
\begin{itemize}
    \item La caratterizzazione di un ambiente è un tema molto importante nel Sound Design.
    
    \item Ogni ambiente ha il suo tipico riverbero ambientale.
    
    \item Modellare correttamente un ambiente, in termini di riverbero, molto spesso è fondamentale al fine di avere un sound design musicale ed efficace.
    
    \item Possiamo distinguere differenti «componenti» di un segnale riverberato:
    \begin{itemize}
        \item Suono diretto (Dry)
        \item Early Reflections
        \item Late riverberation
        \item Decay Rate
        \item Predelay (T0)
    \end{itemize}
\end{itemize}

\section{Reverberation Time}

\begin{itemize}
    \item Il Reverberation Time misura il cosiddetto $T_{60}$, cioè il tempo dopo il quale il segnale di riverbero è attenuato di 60 dB rispetto al suo livello di partenza.
    
    \item Il $T_{60}$, in natura, è funzione di differenti fattori:
    \begin{itemize}
        \item Dimensioni $W \times L \times H$ dell’ambiente
        \item Materiali costruttivi delle pareti (riflettenti e non)
        \item Elementi acustici ambientali
    \end{itemize}
\end{itemize}

\section{Effetto del pre-delay}
\begin{itemize}
    \item Il pre-delay tra il segnale diretto e la sua replica ha i seguenti effetti dipendenti dal tempo:
    \begin{itemize}
        \item $< 50$ ms: Haas effect, o doubling
        \item $50 < \text{delay} < 100$ ms: near-echo (prolonged/indistinct)
        \item $> 100$ ms: echo
    \end{itemize}
\end{itemize}


\section*{Artificial Reverberation}

\begin{itemize}
    \item Un campo di lunga durata:
    \begin{itemize}
        \item Anni '20: spazio acustico vuoto con altoparlante e microfono, riverbero a molla Hammond.
        \item Anni '50: riverbero a piastra.
        \item 1960s: BBD (Panasonic MN3011 ha taps di uscita multipli in posizioni reciprocamente prime).
        \item Schroeder (Bell Labs) introdusse il "Colorless Artificial Reverberation" nel 1961 nell'IRE Trans. on Audio, basato su linee di ritardo e filtri all-pass. Moorer aggiunse un filtro a un polo per il decadimento in funzione della frequenza.
        \item Negli anni '70 furono commercializzati diversi riverberi digitali, tra cui il Lexicon Delta T-101 e il Lexicon 224.
        \item In seguito sono stati introdotti altri metodi: image-source method (Allen and Berkeley, 1979), Digital waveguides (Smith, 1985), etc.
    \end{itemize}
    
    \item Esistono diversi approcci:
    \begin{itemize}
        \item Digital waveguide mesh
        \item Numerical simulations
        \item Convolution with measured IR
        \item Delay network methods
        \item Schroeder
        \item Dattorro
    \end{itemize}
\end{itemize}

\section{Riverbero a Convoluzione}

\begin{itemize}
    \item La stanza come LTI --> la sua IR caratterizza perfettamente la stanza (Hp: campo riverberante).
    \item Se registriamo l'IR, possiamo fare la convoluzione di qualsiasi suono con esso per ottenere il riverbero atteso.
\end{itemize}
\[
    y_{\text{rev}}[n] = \sum_{m=0}^{M} h[m] \cdot x[n - m]
\]


\begin{itemize}
    \item L'IR è una proprietà dell'intero spazio solo se il campo diretto è trascurabile (condizione di campo riverberante) --> ci sarà sempre un volume (eventualmente grande) dello spazio in cui il campo diretto non è trascurabile.
    \item Quando il campo diretto non è trascurabile, l'IR dipende dalla distanza e dalla posizione della sorgente e del ricevitore.
\end{itemize}


\begin{itemize}
    \item L'implementazione diretta è costosa:
    \begin{itemize}
        \item $L$ = lunghezza del segnale di ingresso --> la convoluzione richiede $L$ MUL e $(L-1)$ SUM per ogni campione di uscita.
        \item Se $M$ = lunghezza del segnale in ingresso, il segnale in uscita sarà lungo $L+M$ campioni.
    \end{itemize}
    \item Convoluzione nel tempo = Prodotto in frequenza:
    \begin{itemize}
        \item Se $K$ = bin della FFT = dimensione del frame si ha:
        \item Costo della FFT: $O(N \log N)$
        \item $K$ prodotti complessi (4K MUL + 3K SUM)
        \item Costo dell'IFFT
    \end{itemize}
    \item Costo computazionale ridotto per $K$ elevati, ma latenza elevata!
    \item Sono stati proposti molti algoritmi specializzati.
\end{itemize}

\section{Riverbero a Convoluzione (Vantaggi e Svantaggi)}

\begin{itemize}
    \item \textbf{Vantaggi:}
    \begin{itemize}
        \item Istantanea perfetta di una stanza in campo riverberante.
        \item Concettualmente facile da capire.
    \end{itemize}
    
    \item \textbf{Svantaggi:}
    \begin{itemize}
        \item È necessario effettuare molte misure di IR con diverse posizioni di sorgente e ricevitore, selezionandone una.
        \item Impossibile cambiare le proprietà della stanza: l'IR è un'istantanea della stanza (cfr. campionamento).
        \item Alto costo computazionale (+ compromesso con la latenza).
        \item Complesso da ottenere, sono necessari altoparlanti e microfoni costosi.
    \end{itemize}
\end{itemize}


\section{Schroeder reverb}
\textbf{• Un feedback comb che emula gli echi ripetuti, che è ciò che ci interessa.}
\textbf{• Tuttavia, colora anche lo spettro}
\textbf{• Una topologia che introduce echi senza colore spettrale è il ritardo allpass di Schroeder:}

\section{Feedback Delay Network (FDN)}

\begin{itemize}
    \item Introdotte da Gerzon nel 1971-72.
    \item Parte dall'osservazione che i singoli filtri comb danno una scarsa qualità delle code di riverbero (colorano troppo il segnale se sono troppi pochi).
    \item Più filtri comb suonano molto bene però quando sono accoppiati, di fatto quando vengono combinati e messi in cross-feedback.
\end{itemize}

\begin{itemize}
    \item La matrice di ricombinazione deve avere la proprietà di ortogonalità.
    \item Si sfruttano le matrici di Hadamard.
    \[
H_1 = [1]
\]

\[
H_2 = 
\begin{bmatrix}
1 & 1 \\
1 & -1
\end{bmatrix}
\]

\[
H_{2^k} = 
\begin{bmatrix}
H_{2^{k-1}} & H_{2^{k-1}} \\
H_{2^{k-1}} & -H_{2^{k-1}}
\end{bmatrix}
= H_2 \otimes H_{2^{k-1}}
\]

    \item Alcune modifiche alla matrice di Hadamard 4x4 sono state proposte da Stautner e Puckette.
\end{itemize}

\section{Feedback Delay Network (FDN) (Approccio ``lossless'')}

\begin{itemize}
    \item Un approccio ``lossless" per la progettazione di matrici di feedback:
    \begin{itemize}
        \item \textbf{IDEA:} un late reverb IR dovrebbe assomigliare a un rumore a decadimento esponenziale (cfr. con velvet noise).
        \item Un buon approccio progettuale consiste nel partire da un prototipo senza perdita (T60 infinito) e lavorare per renderlo un buon generatore di rumore.
        \item Regolare i ritardi (mutuamente primi).
        \item Esistono diversi metodi per ottenere una mode density uguale.
        \item Regolare la matrice di feedback.
        \item Una volta ascoltato un rumore omogeneo, i ritardi possono essere regolati per ottenere il giusto tempo di decadimento per ogni banda.
    \end{itemize}
\end{itemize}
